\begin{frame}{What about privacy: GDPR and ePrivacy Directive}
  Since browser fingerprinting relies on the collection of personal data, companies using this technique must comply with the strict requirements of the GDPR and the ePrivacy Directive:
  \begin{itemize}
    \item the data collection process must be transparent to users
    \item companies must ask for consent when personal data is involved
  \end{itemize}
  Otherwise, companies often have proprietary methods to perform browser fingerprinting and those often are hard to be detected and do not use personal data.
\end{frame}

\begin{frame}{Conclusions: prevention and mitigation}
  \begin{itemize}
    \item use privacy-focused browsers like Tor, Mullvad and anti-fingerprinting extensions like Privacy-Badger and uBlock
          \vspace{0.5cm}
    \item disable unnecessary plugins, disable useless extensions and consider using default settings to blend in with a larger crowd
          \vspace{0.5cm}
    \item keep the software updated
          \vspace{0.5cm}
    \item if you are a developer, build your own anti-fingerprinting countermeasure
  \end{itemize}
\end{frame}