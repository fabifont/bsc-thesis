\begin{frame}{Why it is used: the constructive way}
  \begin{itemize}
    \item combat fraud or credential hijacking by checking that a user who logs into a specific site is likely the legitimate user
    \item suggesting updates whether a device is out of date and verify that it is genuine and known to the system
    \item bot detection: generic or missing fingerprints, consistency across sessions and behavioral analysis
  \end{itemize}
\end{frame}

\begin{frame}{Why it is used: the questionable way}
  \begin{itemize}
    \item track users across websites and collect information about their habits and their tastes without the users knowing about it:
      \begin{itemize}
        \item \textbf{Fingerprints as Global Identifiers:} if a device has a unique fingerprint, it is akin to a cookie that cannot be deleted
        \item \textbf{Fingerprint + IP address}
          \begin{itemize}
            \item \textbf{as Cookie Regenerators:} since users often retain the same IP for extended period their deleted cookies can be restored by matching the IP and the fingerprint
            \item \textbf{in the absence of Cookies:} unmask different machines behind the same IP
          \end{itemize}
      \end{itemize}
  \end{itemize}
\end{frame}

\begin{frame}{Why it is used: advertising and customizing the online experience}
  \begin{itemize}
  \item data collected through browser fingerprinting methods allows advertising businesses to create a custom profile for targeted advertising which indirectly means higher revenue for the company
  \item adapt content and prices due to your habits, status and location
  \end{itemize}
\end{frame}

\begin{frame}{Why it is used: the destructive way}
  Deliver exploits that are tailored for a specific user configuration
\end{frame}
